\documentclass{article}
\usepackage{graphicx} % Required for inserting images

\title{TITIT title}
\author{name name name }
\date{March 2024}

\begin{document}

\maketitle

\section{Introduction}

\subsection{How Does a Random Forest Work?}

\begin{itemize}
    \item \textbf{Bootstrap Sampling:} Starts by creating multiple samples from the original dataset through bootstrapping, selecting random data points with replacement.
    \item \textbf{Building Many Decision Trees:} For each sample, a decision tree is built, considering only a random subset of features at every split point.
    \item \textbf{Making Predictions:}
    \begin{itemize}
        \item Classification: For classification, each tree votes for a class, and the majority vote is the final prediction.
        \item Regression: For regression, the predictions of the trees are averaged to produce the final output.
    \end{itemize}
\end{itemize}

\textbf{Linear Transformation:} A linear transformation is a function from one vector space to another that respects the underlying (linear) structure of each vector space. A linear transformation is also known as a linear operator or map.
\\\\
\begin{tabular}{c c}
    Cell1 & Cell2 \\
    Cell3 & Cell4 \\
\end{tabular}
\\ \\
\begin{tabular}{|l|c|r|} % The '|' in the column specification adds vertical lines
\hline % This adds a horizontal line at the top of the table
Column 1 & Column 2 & Column 3 \\ \hline % End of row and horizontal line
Row 1 & Data 1 & Data 2 \\ \hline
Row 2 & Data 3 & Data 4 \\ \hline
% \hline at the end adds a horizontal line at the bottom of the table
\end{tabular}
\\\\

\subsection{There are so many symbols}

$\nabla$
\\
$\infty$

\subsection{and the greek alphabet}

alpha $\alpha$
\\
beta $\beta$
\\
gamma $\gamma$
\\
delta $\delta$
\\
epsilon $\epsilon$
\\
epsilon $\zeta$
\\
eta $\eta$
\\
theta $\theta$
\\
iota $\iota$
\\
kappa $\kappa$
\\
lambda $\lambda$
\\
mu $\mu$
\\
nu	$\nu$
\\
xi	$\xi$
\\
omicron	$\omicron$
\\
pi	$\pi$
\\
rho	$\rho$
\\
sigma (final form)	$\varsigma$
\\
sigma	$\sigma$
\\
tau	$\tau$
\\
upsilon	$\upsilon$
\\
phi	$\phi$
\\
chi	$\chi$
\\
psi	$\psi$
\\
omega	$\omega$

\end{document}

\end{document}