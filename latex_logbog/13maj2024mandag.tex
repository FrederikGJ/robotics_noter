\documentclass{article}
\usepackage{graphicx} % Required for inserting images

\title{Noter mandag 13. maj 2024}
\author{Frederik GJ}
\date{Maj 2024}

\begin{document}

\maketitle

\section{Vi lærer om at programmere NodeMCU og Arduino}

\begin{itemize}
    \item \textbf{Logbog for dagen:} Tobias holdt oplæg om programmering i C og C++ som skal bruges til at programmere NodeMCU og Arduino. 
    \item \textbf{Gruppearbejde:} Vi har fokuseret på undervisningen i programmering. 
\end{itemize}

\section{Hvad har jeg lært i dag?}
\begin{itemize}
    \item Man kan bruge ArduinoIDE eller VS Code til at progammere robotterne. 
    \item Vi skal programmere asynkront, da der kun er en CPU kerne på NodeMCU'en. 
    \item Vi kan bruge et fumlebræt (Bread Board) når vi programmerer vores NodeMCU. 
    \item Vi bruger primært hukommelse på Stack. Og vi får altså umiddelbart ikke brug for at allokere hukommelse på Heap. 
    \item Vi srkiver i programmeringssproget C. Vi gennemgår også hvordan man laver en klasse med C++. Så vi kan lave mere objektorienteret programmering. I C++ har vi både en header fil (toto.h) og en cpp fil (toto.cpp). 
    \item Tobias opfordrer os til at google os frem. Da der ligger mange gode toutorials på nettet omkring NodeMCU
    \item man skal have sit eget micro usb kabel med. Sådan at man kan forbinde med vores NodeMCU. 
    
\end{itemize}

\section{Vores robotbil}
\begin{itemize}
    \item I dag har fokus været på hvordan vi kan programmere vores NodeMCU, som er forudsætningen for at man kan køre med bilen. 
\end{itemize}


\end{document}